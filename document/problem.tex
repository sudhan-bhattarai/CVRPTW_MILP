\documentclass[11pt]{article}
\usepackage{amsmath, amsfonts}
\usepackage[margin=1in]{geometry}
\usepackage{booktabs}
\usepackage{setspace}
\doublespacing

\title{Vehicle Routing Problem as a Mixed Integer Linear Program}
\author{Sudhan Bhattarai}
\date{}

\begin{document}
\maketitle
\section{Introduction} \label{sec:introduction}
The Vehicle Routing Problem (VRP) is a well-known optimization problem in supply chain and logistics management. A standard approach to model this problem is through a Mixed-Integer Linear Program (MILP). This repository presents a simple example of formulating and solving the VRP using MILP techniques.

\subsection{Problem Overview} \label{sec:overview}
We consider a distribution center that operates a fleet of vehicles, all identical in size and capacity. A set of customers, each located at a distinct geographic location, has known commodity demands. Each customer specifies a preferred \emph{time window} during which deliveries must occur. That is, a vehicle must arrive within the specified window to make a delivery.

Upon arrival, each delivery requires a fixed amount of service time. For instance, if a customer specifies a delivery window of 2:00–3:00 PM, the vehicle must arrive no earlier than 2:00 PM and no later than 3:00 PM.

The objective is to determine vehicle routes that minimize the total travel distance while ensuring that:
\begin{itemize}
    \item All customer demands are met,
    \item Each vehicle operates within its capacity, and
    \item All deliveries occur within the respective time windows.
\end{itemize}

\section{Problem Formulation}

To formulate the VRP described in Subsection~\ref{sec:overview}, we begin by defining the relevant notation in Table~\ref{tab:notation}.

\begin{table}[htbp]
\centering
\begin{tabular}{ll}
\toprule
$V$ & Number of vehicles available at the depot \\
$I$ & Number of customers, with the customer set defined as $\{1, \ldots, I\}$ and indexed by $i$ \\
$(x_i, y_i)$ & Geographical coordinates of customer $i \in \{1, \ldots, I\}$ \\
$(x_0, y_0)$ & Geographical coordinates of the depot, denoted as node $i = 0$ \\
$d_{i,j}$ & Distance between nodes $i$ and $j$, for all $i, j \in \{0, 1, \ldots, I\},\ i \neq j$ \\
$q$ & Capacity of each vehicle (identical for all $v = 1, \ldots, V$) \\
$e_i$ & Earliest allowable service start time for customer $i \in \{1, \ldots, I\}$ \\
$l_i$ & Latest allowable service start time for customer $i \in \{1, \ldots, I\}$ \\
$\delta$ & Fixed service duration at each customer location \\
$M$ & A sufficiently large constant used for time-based constraints \\
\bottomrule
\end{tabular}
\caption{Model Notation}
\label{tab:notation}
\end{table}

\subsection{Decision Variables}

Based on the notation in Table~\ref{tab:notation}, we define the decision variables of the MILP model in Table~\ref{tab:decision_vars}.

\begin{table}[htbp]
\centering
\begin{tabular}{ll}
\toprule
$x_{i,j} \in \{0,1\}$ & 1 if a vehicle travels from $i = 0, 1, \cdots, I$ to $j = 0, 1, \cdots, I: j \neq i$, 0 o/w \\
$y_i \geq 0$ & Load of the vehicle just before servicing customer $i = 1, \cdots, I$ \\
$s_i \geq 0$ & Start time of service at customer $i = 1, \cdots, I$ \\
\bottomrule
\end{tabular}
\caption{Decision Variables}
\label{tab:decision_vars}
\end{table}

This formulation assumes a \emph{homogeneous fleet} in which all vehicles have the same capacity and characteristics. In the case of a \emph{heterogeneous fleet}, the load variable $y_i$ would need to be vehicle-specific (e.g., $y_{i,v}$) to capture varying capacities and routing choices per vehicle. The binary decision variables can also be modified accordingly.

\subsection{MILP Model}

The MILP formulation for the VRP is given by equations in Model~\eqref{milp_model}.

\begin{subequations}
\label{milp_model}
\begin{align}
    \min \quad & \sum_{i = 0}^I \sum_{\substack{j = 0 \\ j \neq i}}^I d_{i, j} x_{i, j} \label{obj} \\
    \text{s.t.} \quad
    & \sum_{i = 1}^I x_{0, i} \leq V, \label{c1} \\
    & \sum_{i = 1}^I x_{i, 0} = \sum_{i = 1}^I x_{0, i}, \label{c2} \\
    & \sum_{\substack{i = 0 \\ i \neq j}}^I x_{i, j} = 1, \quad \forall j \in \{1, \ldots, I\}, \label{c3} \\
    & \sum_{\substack{j = 0 \\ j \neq i}}^I x_{i, j} = 1, \quad \forall i \in \{1, \ldots, I\}, \label{c4} \\
    & s_j \geq s_i + t_{i, j} x_{i, j} + \delta - M (1 - x_{i, j}), \quad \forall i \neq j, \label{c5} \\
    & y_j \geq y_i + d_i x_{i, j} - M (1 - x_{i, j}), \quad \forall i \neq j, \label{c6} \\
    & y_i \leq q, \quad \forall i \in \{0, 1, \ldots, I\}, \label{c7} \\
    & e_i \leq s_i \leq l_i, \quad \forall i \in \{1, \ldots, I\}, \label{c8} \\
    & y_i \geq 0,\ s_i \geq 0, \quad \forall i \in \{1, \ldots, I\}, \label{c9} \\
    & x_{i, j} \in \{0, 1\}, \quad \forall i, j \in \{0, \ldots, I\},\ i \neq j. \label{c10}
\end{align}
\end{subequations}

The equations in the MILP model~\eqref{milp_model} serve the following purposes:

\begin{itemize}
    \item \textbf{The objective function~\eqref{obj}} minimizes the total travel distance of all vehicles.
    \item \textbf{Constraint~\eqref{c1}}: Limits the number of vehicles that can leave the depot to at most $V$, the total fleet size.
    \item \textbf{Constraint~\eqref{c2}}: Ensures that every vehicle that departs from the depot eventually returns to it, maintaining route continuity.
    \item \textbf{Constraint~\eqref{c3}}: Guarantees that each customer is visited exactly once by one vehicle.
    \item \textbf{Constraint~\eqref{c4}}: Ensures that after servicing a customer, the vehicle continues its route to the next location (except for the last node on a route which is the depot itself).
    \item \textbf{Constraint~\eqref{c5}}: Enforces correct sequencing of service times. If a vehicle travels directly from customer $i$ to $j$, then service at $j$ must start after finishing service at $i$ and traveling to $j$. The big-$M$ term deactivates the constraint when arc $(i,j)$ is not used.
    \item \textbf{Constraint~\eqref{c6}}: Tracks cumulative demand served along the route. If a vehicle moves from $i$ to $j$, the load at $j$ must account for the demand already served at $i$.
    \item \textbf{Constraint~\eqref{c7}}: Ensures that the cumulative demand on any vehicle does not exceed its capacity $q$.
    \item \textbf{Constraint~\eqref{c8}}: Enforces that service at each customer starts within their preferred time window, between $e_i$ and $l_i$.
    \item \textbf{Constraint~\eqref{c9}}: Declares the non-negativity of the continuous decision variables $y_i$ and $s_i$.
    \item \textbf{Constraint~\eqref{c10}}: Defines the binary nature of routing decisions: $x_{i,j}$ equals 1 if a vehicle travels from $i$ to $j$, and 0 otherwise.
\end{itemize}
\end{document}